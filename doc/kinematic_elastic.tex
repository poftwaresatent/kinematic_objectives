\documentclass{article}

\usepackage{amsmath}
\usepackage{paralist}

\begin{document}

\title{Kinematic Elastic}
\author{Roland Philipsen}
\maketitle



\section{Background}

We trace the elastic band concept back to~\cite{quinlan:1994} (pioneering work formulated in configuration space) and~\cite{brock:1999} (operational space formulation).
Related approaches include CHOMP~\cite{ratliff:2009} and STOMP~\cite{kalakrishnan:2011}.
A thorough theoretical analysis of the kinematic and algorithmic singularities that are challenging to the methods proposed here can be found in~\cite{chiaverini:1997}.



\section{Preliminaries}

$dq$ is a displacement, velocity, or acceleration in joint space $\mathcal{C}$.
$dx$ is a displacement, velocity, or acceleration in task space.
Frequently, task space will be a (subspace of) workspace $\mathcal{W}$.

Several variants are available for computing pseudo-inverses, but here all of them are just written $\cdot^\#$.
A simple and surprisingly robust method is to use an SVD-based Moore-Penrose scheme with a fixed Eigenvalue threshold.
Warnings against this approach are common in the literature, but it seems to work, and using more advanced schemes like damped or adaptively damped pseudo-inverses introduce parameters that need to be tuned.
In addition, a particularly tricky issue with damping arises in the context of generalized task prioritization:
a less-damped lower priority task can easily interfere with a higher-priority damped task.
In practical cases, this happens a lot, because there tend to be singularilty-free tasks at lower priority levels, and singularity-prone tasks at higher priority levels.
The idea of simply carrying the damping factor from higher priority tasks all the way down does not pan out, because \emph{(i)} the different tasks have different distance metrics associated with them and \emph{(ii)} the ``nullspace'' projectors produced from damped matrices leave residuals in the directions that they should kick out.



\section{Strict Prioritization Scheme}

\begin{align}
  dq &= J_1^\#dx_1 + N_1 ( J_2^\#dx_2 + N_2 ( \cdots )) \\
  &= \sum_{i=1}^T \left( \prod_{j=0}^{i-1}N_j \right) J_i^\#dx_i\\
  N_0 &= I \\
  N_i &= I - J_i^\#J_i
\end{align}

According to~\cite{mistry:2007} this is due to~\cite{chiaverini:1997}.
Advantage: proper nullspace projection.
Disadvantage: convergence can be slow\ldots extremely slow.



\section{Sloppy Prioritization Scheme}

\begin{align}
  dq_1 &= 0 \\
  P_1 &= I \\
  \bar{J}_i &= J_iP_{i-1} \\
  dq_i &= dq_{i-i} + \bar{J}_i^\#(dx_i-J_idq_{i-1}) \\
  P_i &= P_{i-1} - \bar{J}_i^\#\bar{J}_i
\end{align}

Introduced by~\cite{siciliano:1991}.
Advantage: good convergence (unless we hit singularities, of course).
Disadvantage: lower-priority tasks can interfere with higher-priority tasks due to ``algorithmic singularities'' (when $\bar{J}_i$ becomes singular along directions that $P_{i-1}$ would like to kick out).

BTW~\cite{mistry:2007} show that the above recursive formulation is equivalent to simply stacking all tasks like this:

\begin{equation}
  dq =
  \begin{bmatrix}
    J_1 \\
    J_2 \\
    \vdots
  \end{bmatrix}
  ^\#
  \begin{bmatrix}
    dx_1 \\
    dx_2 \\
    \vdots
  \end{bmatrix}
\end{equation}



\section{Constraints, Tasks, and Objectives}

All entities are represented as tuples $(dx_i, J_i)$ where $dx$ defines how much they want to move within their task space, and $J_i$ is (the linearization of) the mapping from joint space to task space.

\begin{compactdesc}
\item[Constraints]
  $dx$ is interpreted as a displacement required to achieve constraint satisfaction.
  After computing $dq$ from all constraints, the waypoint's model state is reset to the conforming positions.
  Its velocities are also reset to something consistent with the constraints.
\item[Tasks]
  $dx$ is interpreted as a desired acceleration towards achieving the task.
  Typically, we'll use a feedback control scheme (i.e.\ PD control) to compute $dx$ from the waypoint's current position and velocity.
  Tasks are accumulated using a prioritization scheme with an initial projector $P_1$ that reflects the constraints.
  This is inspired by the scheme in~\cite{baerlocher:2001}.
\item[Objectives]
  Similarly to tasks, $dx$ is interpreted as an acceleration.
  However, objectives are not subjected to prioritization.
  Instead, their desired accelerations are simply summed up in joint space.
  By projecting the resulting \emph{bias acceleration} into the nullspace of the constraints and tasks, the overall effect is similar to virtual operational space forces.
\end{compactdesc}



\footnotesize
\bibliographystyle{plain}
\bibliography{kinematic_elastic}

\end{document}
